%%%%%%%%%%%%%%%%%%%%%%%%%%%%%%%%%%%%%%%%%%%%%%%%%%%%%%%%%%%%%%%%%%%%%%
% writeLaTeX Example: Academic Paper Template
%
% Source: http://www.writelatex.com
% 
% Feel free to distribute this example, but please keep the referral
% to writelatex.com
% 
%%%%%%%%%%%%%%%%%%%%%%%%%%%%%%%%%%%%%%%%%%%%%%%%%%%%%%%%%%%%%%%%%%%%%%
% How to use writeLaTeX: 
%
% You edit the source code here on the left, and the preview on the
% right shows you the result within a few seconds.
%
% Bookmark this page and share the URL with your co-authors. They can
% edit at the same time!
%
% You can upload figures, bibliographies, custom classes and
% styles using the files menu.
%
% If you're new to LaTeX, the wikibook is a great place to start:
% http://en.wikibooks.org/wiki/LaTeX
%
%%%%%%%%%%%%%%%%%%%%%%%%%%%%%%%%%%%%%%%%%%%%%%%%%%%%%%%%%%%%%%%%%%%%%%
\documentclass[twocolumn,showpacs,%
  nofootinbib,aps,superscriptaddress,%
  eqsecnum,prd,notitlepage,showkeys,10pt]{revtex4-1}

\usepackage[portuguese]{babel}
\usepackage[utf8]{inputenc}
\usepackage{amssymb}
\usepackage{amsmath}
\usepackage{graphicx}
\usepackage{dcolumn}
\usepackage{xcolor}
\usepackage[pdfstartview=FitH,
            colorlinks,
            bookmarksnumbered,
            bookmarksopen,
            linktocpage,
            urlcolor=blue,
            linkcolor=cyan,
            citecolor=red!70!black]{hyperref}

\usepackage{listings}
\usepackage{color}

\definecolor{dkgreen}{rgb}{0,0.6,0}
\definecolor{gray}{rgb}{0.5,0.5,0.5}
\definecolor{mauve}{rgb}{0.58,0,0.82}

\lstset{frame=tb,
  language=[90]Fortran,
  aboveskip=3mm,
  belowskip=3mm,
  showstringspaces=false,
  columns=flexible,
  basicstyle={\small\ttfamily},
  numbers=none,
  numberstyle=\tiny\color{gray},
  keywordstyle=\color{blue},
  commentstyle=\color{dkgreen},
  stringstyle=\color{mauve},
  breaklines=true,
  breakatwhitespace=true,
  tabsize=4
}

\usepackage[nottoc,numbib]{tocbibind}

\usepackage[cmintegrals,cmbraces]{newtxmath}
\usepackage{ebgaramond-maths}

\makeatletter
  \DeclareSymbolFont{ntxletters}{OML}{ntxmi}{m}{it}
  \SetSymbolFont{ntxletters}{bold}{OML}{ntxmi}{b}{it}
  \re@DeclareMathSymbol{\leftharpoonup}{\mathrel}{ntxletters}{"28}
  \re@DeclareMathSymbol{\leftharpoondown}{\mathrel}{ntxletters}{"29}
  \re@DeclareMathSymbol{\rightharpoonup}{\mathrel}{ntxletters}{"2A}
  \re@DeclareMathSymbol{\rightharpoondown}{\mathrel}{ntxletters}{"2B}
  \re@DeclareMathSymbol{\triangleleft}{\mathbin}{ntxletters}{"2F}
  \re@DeclareMathSymbol{\triangleright}{\mathbin}{ntxletters}{"2E}
  \re@DeclareMathSymbol{\partial}{\mathord}{ntxletters}{"40}
  \re@DeclareMathSymbol{\flat}{\mathord}{ntxletters}{"5B}
  \re@DeclareMathSymbol{\natural}{\mathord}{ntxletters}{"5C}
  \re@DeclareMathSymbol{\star}{\mathbin}{ntxletters}{"3F}
  \re@DeclareMathSymbol{\smile}{\mathrel}{ntxletters}{"5E}
  \re@DeclareMathSymbol{\frown}{\mathrel}{ntxletters}{"5F}
  \re@DeclareMathSymbol{\sharp}{\mathord}{ntxletters}{"5D}
  \re@DeclareMathAccent{\vec}{\mathord}{ntxletters}{"7E}
\makeatother

\DeclareMathAlphabet\mathbfcal{OMS}{cmsy}{b}{n}

\begin{document}

\title{
Problemas de Sturm-Liouville \\
Uma abordagem numérica
}
\author{Caio Tomás de Paula}
\affiliation{Departamento de Matemática, Universidade de Brasília, Brasil.}

\begin{abstract}
    Relatório entregue como parte da avaliação do curso de Introdução a
    Métodos Computacionais em Equações Diferenciais Ordinárias (IMCEDO),
    ministrado pelo prof. Dr. Yuri Dumaresq Sobral no segundo semestre letivo
    de 2022 da Universidade de Brasília.
\end{abstract}

\maketitle

\section{Introdução}

explicar o que é o problema, de onde ele vem, porque é importante...

mostrar a forma geral do problema (pegar a eq de Helmholtz, talvez?)

dar exemplos de problemas regulares e irregulares que aparecem em MMF1

Usamos \cite{Sturm-Liouville} como referência principal para o trabalho, além
das notas de aula \cite{notas-aula-IMCEDO}.

\section{Resultados}
\label{sec:resultados}

\subsection{O método numérico}

Ideia do método: transformar o problema de SL, que é um PVC, em um PVI
de modo que resolver o PVC é encontrar a solução do PVI e os zeros da solução
do PVI.

Teo. 186 - 188 (p.304-305): 
garantem que podemos usar a bisseção e NR para encontrar os zeros, i.e.,
os autovalores e as autofunções (com convergência uniforme neste último caso) do problema de SL regular

Além do problema de SL regular, temos duas categorias de problema de
SL singular (as duas aparecem, por exemplo, ao resolver a equação
de Laplace no disco -- \textcolor{red}{conferir essa info})

Teo. 190 - 192 (p.319-321):
garantem que podemos usar a bisseção e NR para encontrar os zeros, i.e.,
os autovalores e as autofunções (com convergência uniforme neste último caso) do problema de SL singular do primeiro tipo

Teo. 194 - 196 (p.337-339):
garantem que podemos usar a bisseção e NR para encontrar os zeros, i.e.,
os autovalores e as autofunções (com convergência uniforme neste último caso) do problema de SL singular do segundo tipo

A aplicação do método para problemas de S-L difere um pouco da aplicação ``tradicional'':
para os problemas de S-L, a cada passo precisamos
%
\begin{enumerate}
    \item propor uma estimativa $\lambda_n^i$ para o autovalor $\lambda_n$; 
    \item resolver o PVI em $u(x, \lambda_n^i)$ dado pelo problema de S-L quando substituímos a condição de
    contorno em 1 pela condição inicial $u'(0) = 1$;
    \item avaliar o quão longe de $0$ está $u(1,\lambda_n^i)$:
    %
    \begin{enumerate}
        \item se muito longe, usamos um método iterativo para obter um próximo chute $\lambda_n^{i+1}$
        e repetimos o processo;
        \item se perto o bastante, paramos.
    \end{enumerate}
    %
\end{enumerate}
%

%
\subsection{Validação do método}\label{subsec:validacao}
%



%
\subsection{Aplicando o método}\label{subsec:aplicacao}
%



%
\subsection{Conclusão}\label{subsec:conclusao}
%

% \lstinputlisting[language=Fortran, caption=Teste]{teste_programa.f90}

\begin{acknowledgments}

We thank\dots

\end{acknowledgments}

% \newpage
% \nocite{*}
% \bibliographystyle{apalike}
\bibliographystyle{siam}
\bibliography{referencias}

\end{document}